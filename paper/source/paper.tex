\documentclass[11pt,preprint, authoryear]{elsarticle}

\usepackage{lmodern}
%%%% My spacing
\usepackage{setspace}
\setstretch{1.2}
\DeclareMathSizes{12}{14}{10}{10}

% Wrap around which gives all figures included the [H] command, or places it "here". This can be tedious to code in Rmarkdown.
\usepackage{float}
\let\origfigure\figure
\let\endorigfigure\endfigure
\renewenvironment{figure}[1][2] {
    \expandafter\origfigure\expandafter[H]
} {
    \endorigfigure
}

\let\origtable\table
\let\endorigtable\endtable
\renewenvironment{table}[1][2] {
    \expandafter\origtable\expandafter[H]
} {
    \endorigtable
}


\usepackage{ifxetex,ifluatex}
\usepackage{fixltx2e} % provides \textsubscript
\ifnum 0\ifxetex 1\fi\ifluatex 1\fi=0 % if pdftex
  \usepackage[T1]{fontenc}
  \usepackage[utf8]{inputenc}
\else % if luatex or xelatex
  \ifxetex
    \usepackage{mathspec}
    \usepackage{xltxtra,xunicode}
  \else
    \usepackage{fontspec}
  \fi
  \defaultfontfeatures{Mapping=tex-text,Scale=MatchLowercase}
  \newcommand{\euro}{€}
\fi

\usepackage{amssymb, amsmath, amsthm, amsfonts}

\def\bibsection{\section*{References}} %%% Make "References" appear before bibliography


\usepackage[round]{natbib}
\bibliographystyle{plainnat}

\usepackage{longtable}
\usepackage[margin=2.3cm,bottom=2cm,top=2.5cm, includefoot]{geometry}
\usepackage{fancyhdr}
\usepackage[bottom, hang, flushmargin]{footmisc}
\usepackage{graphicx}
\numberwithin{equation}{section}
\numberwithin{figure}{section}
\numberwithin{table}{section}
\setlength{\parindent}{0cm}
\setlength{\parskip}{1.3ex plus 0.5ex minus 0.3ex}
\usepackage{textcomp}
\renewcommand{\headrulewidth}{0.2pt}
\renewcommand{\footrulewidth}{0.3pt}

\usepackage{array}
\newcolumntype{x}[1]{>{\centering\arraybackslash\hspace{0pt}}p{#1}}

%%%%  Remove the "preprint submitted to" part. Don't worry about this either, it just looks better without it:
\journal{Journal of Finance}

 \def\tightlist{} % This allows for subbullets!

\usepackage{hyperref}
\hypersetup{breaklinks=true,
            bookmarks=true,
            colorlinks=true,
            citecolor=blue,
            urlcolor=blue,
            linkcolor=blue,
            pdfborder={0 0 0}}


% The following packages allow huxtable to work:
\usepackage{siunitx}
\usepackage{multirow}
\usepackage{hhline}
\usepackage{calc}
\usepackage{tabularx}
\usepackage{booktabs}
\usepackage{caption}
\usepackage{colortbl}

\urlstyle{same}  % don't use monospace font for urls
\setlength{\parindent}{0pt}
\setlength{\parskip}{6pt plus 2pt minus 1pt}
\setlength{\emergencystretch}{3em}  % prevent overfull lines
\setcounter{secnumdepth}{5}

%%% Use protect on footnotes to avoid problems with footnotes in titles
\let\rmarkdownfootnote\footnote%
\def\footnote{\protect\rmarkdownfootnote}
\IfFileExists{upquote.sty}{\usepackage{upquote}}{}

%%% Include extra packages specified by user
% Insert custom packages here as follows
% \usepackage{tikz}

%%% Hard setting column skips for reports - this ensures greater consistency and control over the length settings in the document.
%% page layout
%% paragraphs
\setlength{\baselineskip}{12pt plus 0pt minus 0pt}
\setlength{\parskip}{12pt plus 0pt minus 0pt}
\setlength{\parindent}{0pt plus 0pt minus 0pt}
%% floats
\setlength{\floatsep}{12pt plus 0 pt minus 0pt}
\setlength{\textfloatsep}{20pt plus 0pt minus 0pt}
\setlength{\intextsep}{14pt plus 0pt minus 0pt}
\setlength{\dbltextfloatsep}{20pt plus 0pt minus 0pt}
\setlength{\dblfloatsep}{14pt plus 0pt minus 0pt}
%% maths
\setlength{\abovedisplayskip}{12pt plus 0pt minus 0pt}
\setlength{\belowdisplayskip}{12pt plus 0pt minus 0pt}
%% lists
\setlength{\topsep}{10pt plus 0pt minus 0pt}
\setlength{\partopsep}{3pt plus 0pt minus 0pt}
\setlength{\itemsep}{5pt plus 0pt minus 0pt}
\setlength{\labelsep}{8mm plus 0mm minus 0mm}
\setlength{\parsep}{\the\parskip}
\setlength{\listparindent}{\the\parindent}
%% verbatim
\setlength{\fboxsep}{5pt plus 0pt minus 0pt}



\begin{document}

\begin{frontmatter}  %

\title{A Data Scientific Approach to Equity Backtesting}

% Set to FALSE if wanting to remove title (for submission)




\author[Add1]{Riaz Arbi}
\ead{}





\address[Add1]{University of Cape Town, South Africa}


\begin{abstract}
\small{
Abstract to be written here. The abstract should not be too long and
should provide the reader with a good understanding what you are writing
about. Academic papers are not like novels where you keep the reader in
suspense. To be effective in getting others to read your paper, be as
open and concise about your findings here as possible. Ideally, upon
reading your abstract, the reader should feel he / she must read your
paper in entirety.
}
\end{abstract}

\vspace{1cm}

\begin{keyword}
\footnotesize{
Equity Backtesting \sep Reproducible Research \sep Event-based
Backtesting \sep R \sep RStudio \\ \vspace{0.3cm}
\textit{JEL classification} 
}
\end{keyword}
\vspace{0.5cm}
\end{frontmatter}


\renewcommand{\contentsname}{Table of Contents}
{\tableofcontents}

%________________________
% Header and Footers
%%%%%%%%%%%%%%%%%%%%%%%%%%%%%%%%%
\pagestyle{fancy}
\chead{}
\rhead{}
\lfoot{}
\rfoot{\footnotesize Page \thepage\\}
\lhead{}
%\rfoot{\footnotesize Page \thepage\ } % "e.g. Page 2"
\cfoot{}

%\setlength\headheight{30pt}
%%%%%%%%%%%%%%%%%%%%%%%%%%%%%%%%%
%________________________

\headsep 35pt % So that header does not go over title




\pagebreak 

\section{\texorpdfstring{Introduction
\label{Introduction}}{Introduction }}\label{introduction}

Intro intro intro

Last paragraph

This paper is organised as follows. Section \ref{Backtest section}
briefly introduces the practice of backtesting of investment strategies
in academia and industry, discusses the replication crisis in the
academic literature, and enumerates common methodological issues that
compromise the validity of research. Section \ref{Reproducible Research}
describes the goals and philosophy of the reproducible research movement
and its links to the open source software movement, with a special focus
on the R statistical programming language, the Tidy approach to data
organization and the use of scripting to facilitate reprducibility and
transparency in scientific research. Section
\ref{Backtester Documentation} introduces a backtesting template that is
roughly compatible with the structure of a research compendium - that
is, a basic minimum viable unit of reproducible research.

\section{\texorpdfstring{Anomalies Research in Theory and
Practice\label{Backtest section}}{Anomalies Research in Theory and Practice}}\label{anomalies-research-in-theory-and-practice}

Investigations into why the returns of some common stocks outperform
others have a long history. Professional bodies of knowledge, often
captured in heuristics or rules of thumb, have existed since at least
the early 20th century. ({\textbf{???}}) collected the body of knowledge
of one approach, value investing, into a single textbook in 1934.

As time has progressed, professionals and academics have sought to test
these heuristics in a rigorous manner. A standard way to test whether a
particular heuristic or rule of thumb may drive returns is to
investigate whether an investor that followed the rule in the past would
have significantly different results from an investor that did not. This
process is known as backtesting. Several foundational papers of
investment management (for instance, ({\textbf{???}}), ({\textbf{???}}))
use backtesting methods as a basis for their findings.

Unfortunately, backtesting is a very difficult operation to achieve
correctly. A backtester has to ensure that their dataset is complete and
free of error, which is difficult to know after the fact. For instance,
the exact date on which the financial results of a company were released
may not be known - end-of-year results are never released on the 31st of
December, since those financials need to be computed and audited after
the fact prior to release. The actual release date may be months later,
and if the information contained in the financials drive returns, the
timing of the impact of the release on the market be simply be
unknowable. This problem (which is just an example of many) can be dealt
with in a variety of ways, all of which require a judgment call from the
researcher.

It is standard practice for a researcher to disclose all the judgment
calls made during the backtest process, but these disclosures are
invariably incomplete. This makes validation by other researchers very
difficult. At the extreme, independent validators may arrive at a
completely different conclusion (Hou, Xue, and Zhang
(\protect\hyperlink{ref-Hou2017}{2017})); with ambiguity on both sides
it is unclear whether the discrepancy is due to differences in
underlying data, prepearation methodologies, mathematical errors or some
other factor unrelated to the question at hand.

The academic discipline of anomalies research (as this area is now
known) has grown into a substantial corpus of contradictory claims,
which have been mirrored in the business of investment management by
competing investment philosopies, each with its own library of
peer-reviewed papers (({\textbf{???}})). While markets may very well
accommodate contradictory drivers of returns it is often difficult to
discern whether differing claims are the result of geniune facts or the
result of methodological differences.

Issues with anamalies research The relationship between the issues and
difficulty replicating

\subsection{Characteristics of an Academic Equity
Backtest}\label{characteristics-of-an-academic-equity-backtest}

Basically rip the guts out of the Replicating Anomalies paper - Poor
documentaiton of data collection and munging - Ambiguity around actual
operations done to organise, clean and drop data - No verifiability of
error - Unable to detect p-hacking - Biases: look-ahead, survivorship,
data mining - No transparency (replication, verifiability) - Backtest
overfitting - de Prado, Bailey et al

\subsection{Commercial and Open Source Backtesting Software
Packages}\label{commercial-and-open-source-backtesting-software-packages}

\begin{itemize}
\tightlist
\item
  Survey: Proprietary, free web, free open source
\item
  Prop are out because not transparent
\item
  Open Source: See zipline, qstrader,
\item
  Poor data provenance (yahoo)
\item
  Absence of fundamental data
\item
  Built for pairs and single stocks, not portfolios
\item
  OOP makes transparency difficult (black box)
\item
  Presupposes good data hygeine \& pre-munging
\end{itemize}

\subsection{The Replication Crisis}\label{the-replication-crisis}

\subsection{Common Biases and Methodological
Errors}\label{common-biases-and-methodological-errors}

\section{\texorpdfstring{Reproducible Research and The R Programming
Language\label{Reproducible Research}}{Reproducible Research and The R Programming Language}}\label{reproducible-research-and-the-r-programming-language}

In recent years, as computational power and storage has become
increasingly cheap and available, there have been persistent calls for a
review of the way that scientific reseach is packaged and presented
(Stodden et al. (\protect\hyperlink{ref-Stodden2013}{2013}), Koenker and
Zeileis (\protect\hyperlink{ref-Koenker2009}{2009})).

Although scientific research is an accretive process, involving
verification and building upon earlier datasets and findings, the
computational work done by researchers is often poorly documented, or
absent (Stodden et al. (\protect\hyperlink{ref-Stodden2013}{2013})).
This introduces the risk that views or principles that are considered to
be scientifically verified may, in fact, be false.

Gaps in documentation allow bad reasoning, calculation errors or
spurious results to creep in to the corpus of knowledge of a discipline
because peers are forced to take it on faith that a researcher has a
proper understanding of underlying mathematical concepts, and that
workings have been thouroughly cross-checked before release.

Worse still, because follow-up replication is difficult to impossible
with poor documentation or data availability, incorrect beliefs can
persist for years, or even decades (Munafò et al.
(\protect\hyperlink{ref-Munafo2017}{2017})). \emph{P-hacking}, or the
selection of data and finessing of results to conform to significance
tests of 5\% or lower, has resulted in a situation where a research
finding is as likely to to be false it is to be true (J. P. A. Ioannidis
(\protect\hyperlink{ref-Ioannidis2005}{2005})).

\subsection{Reproducible Research
Terminology}\label{reproducible-research-terminology}

In response to these critiques, a taxonomy of reproducibility has
emerged, allowing peers to effectively categorize research.

This taxonomy allows us to dinstinguish reveiwability (which assumes
mathematical competence and focuses on reasoning) from reproducibility
(which allow the extension of the review into assessement of
mathematical competence and quality of data).

\begin{longtable}[]{@{}ll@{}}
\caption{Common Terminology in Reproducible Science (Stodden et al.
(\protect\hyperlink{ref-Stodden2013}{2013}))\label{Table1}}\tabularnewline
\toprule
\begin{minipage}[b]{0.18\columnwidth}\raggedright\strut
Term\strut
\end{minipage} & \begin{minipage}[b]{0.61\columnwidth}\raggedright\strut
Description\strut
\end{minipage}\tabularnewline
\midrule
\endfirsthead
\toprule
\begin{minipage}[b]{0.18\columnwidth}\raggedright\strut
Term\strut
\end{minipage} & \begin{minipage}[b]{0.61\columnwidth}\raggedright\strut
Description\strut
\end{minipage}\tabularnewline
\midrule
\endhead
\begin{minipage}[t]{0.18\columnwidth}\raggedright\strut
Confirmable\strut
\end{minipage} & \begin{minipage}[t]{0.61\columnwidth}\raggedright\strut
Main conclusions can be attained independently\strut
\end{minipage}\tabularnewline
\begin{minipage}[t]{0.18\columnwidth}\raggedright\strut
Reviewable\strut
\end{minipage} & \begin{minipage}[t]{0.61\columnwidth}\raggedright\strut
Descriptions of methods can be independently assessed and judged\strut
\end{minipage}\tabularnewline
\begin{minipage}[t]{0.18\columnwidth}\raggedright\strut
Replicable\strut
\end{minipage} & \begin{minipage}[t]{0.61\columnwidth}\raggedright\strut
Tools are made available that would allow duplication of results\strut
\end{minipage}\tabularnewline
\begin{minipage}[t]{0.18\columnwidth}\raggedright\strut
Auditable\strut
\end{minipage} & \begin{minipage}[t]{0.61\columnwidth}\raggedright\strut
Sufficient records exist (perhaps privately) to defend research\strut
\end{minipage}\tabularnewline
\begin{minipage}[t]{0.18\columnwidth}\raggedright\strut
Reproducable\strut
\end{minipage} & \begin{minipage}[t]{0.61\columnwidth}\raggedright\strut
Auditable research is made openly available, including code and
data\strut
\end{minipage}\tabularnewline
\begin{minipage}[t]{0.18\columnwidth}\raggedright\strut
To Verify\strut
\end{minipage} & \begin{minipage}[t]{0.61\columnwidth}\raggedright\strut
To check that computer code correctly performs the intended
operation\strut
\end{minipage}\tabularnewline
\begin{minipage}[t]{0.18\columnwidth}\raggedright\strut
To Validate\strut
\end{minipage} & \begin{minipage}[t]{0.61\columnwidth}\raggedright\strut
To check that the results of a computation agree with observations\strut
\end{minipage}\tabularnewline
\bottomrule
\end{longtable}

This paper uses reproducibility in the sense defined in \ref{Table1},
which is to say, it refers to research output being bundled and
distributed with well documented, fully transparent code and data that
allows a peer to fully review, replicate, audit and thereby confirm
results of a body of work (Stodden et al.
(\protect\hyperlink{ref-Stodden2013}{2013})).

\subsection{Distinguising Between Open Source ands Closed Source
Software}\label{distinguising-between-open-source-ands-closed-source-software}

In general, software is written in text files, by humans, in a
particular programming language. This \emph{source code} is
\emph{compiled} or \emph{interpreted} by another program into binary
machine code (which is not readable by humans) and distributed for
execution. The software for the popular spreasheet program Microsoft
Excel, for instance, would be written by a group of humans in a
programming language. These text files would be compiled by a compiler
program into a binary executable program, which is distributed to
consumers who can execute the binary. When someone clicks the Excel icon
on their computer to `launch' the program, they are executing the
binary.

Open source software is software for which the \emph{source code} is
made publically available. Microsoft Excel is closed source, and the
source code is not made available. Source code can be read by humans,
binary cannot. Changes to source code can be meaningfully interpreted by
humans, because one can read the changes in plain text. Changes to
binary code cannot be interpreted by humans, because the changes are
simply alterations to a very long sequence of \texttt{0} and \texttt{1}
digits. It is trivial to compile source code into binary code. It is
extremely difficult to accurately decompile binary code into source
code, and the tools are not readilty available (({\textbf{???}})).

\subsection{The Relationship between Reproducible Research and Open
Source
Software}\label{the-relationship-between-reproducible-research-and-open-source-software}

Open source software development principles exhibit properties which are
useful to practitioners of reproducible research.

By definition, the text-based nature of source code makes all source
code reproducible (and, by corollary, auditable). The implications of
this are twofold. Firstly, a researcher using open source software for
computation can be assured that the full software stack will be open to
scrutiny. That is, at the limit, an auditor could inspect every single
line of code from some arbitrary `open starting point' up to verify that
the computations are correct. This `open starting point' is as the
operating systme level for linux based machines, and from the
application level up for Mac and Windows based machines.

Secondly, and perhaps more importantly, the open source software
community has decades of experience in defining good practices for
creating and maintaining large, open, verfiable and repeatable
enviroments. These principles are often quoted in the folkloric `Unix
Way', which manifests in epithets such as ``text is the universal
interface'' (Baum and Sirin (\protect\hyperlink{ref-Baum2002}{2002})),
``each unit should do one thing and do it well'', and ``build things to
talk to other things'' (({\textbf{???}})). Reproducible research
practice is built on the same principles - simple, readable text-based
data and code where possible; clear separation between data, code and
results; complete transparency coverage; and portability by design (see
Baum and Sirin (\protect\hyperlink{ref-Baum2002}{2002}), Bache and
Wickham (\protect\hyperlink{ref-Bache2014}{2014}), Wickham
(\protect\hyperlink{ref-Wickham2014}{2014})).

Historically, there has been close collaboration between the open source
software and the free software communities. This means that, as a rule,
most open source software is free or has a free analogue. The core tools
of Data Science, for instance (such as the python and R programming
languages, the Apache Foundation of big data processing packages and the
Jupyter and RStudio interactive development environemnts), are all open
source and free. This means that the \emph{cost} of replication is not
prohibitive: in general, reproducible research should be reproducible
for free on commodity hardware.

\subsection{The Suitability of R for Reproducible
Research}\label{the-suitability-of-r-for-reproducible-research}

The R programming language was originally conceived as a project to
build out an open source statistical programming environment. In
development since 1997, base R is now a mature ecosystem comprising a
scriptable language, graphical rendering system and debugger (R Core
Team (\protect\hyperlink{ref-RCoreTeam2018}{2018})). A large community
of third party developers (some commercial, most free) have extended
base R with over 13000 packages, including a fully-fledged interactive
development environmnent (Rstudio), interactive dashboarding web server
(Shiny) and bindings to LaTex for simple rendering of publication-ready
LaTex documents from markdown (knitr and rmarkdown) (RStudio Team
(\protect\hyperlink{ref-RStudioTeam2015}{2015})). Because of its roots
in open source, it adheres to many open source software conventions. For
instance-

\begin{itemize}
\tightlist
\item
  All source code is freely available
\item
  All development work in R is done in plain text files, which are
  transparently human and machine readable.
\item
  The R project has adopted the GNU General Public License verison 2,
  which ``does not restrict anyone from making use of the program in a
  specific field of endeavor'' (R Core Team
  (\protect\hyperlink{ref-RCoreTeam2018}{2018})). This free
  avaialability means that anybody with a commodity computer and an
  internet connection can install and use R.
\end{itemize}

Although the reproducible research community is principles-based and
language-agnostic, the free, text-based nature of the R language have
made it a good candidate for reproducible work (({\textbf{???}})).
Marwick, Boettiger, and Mullen
(\protect\hyperlink{ref-Marwick2018}{2018}) review the concept of a
\emph{research compendium}, and explore how research done in R can be
packaged into compendia. They argue that a compendium is defined by
three principles - 1. Files should be organised according to a
prevailing standard so that others can immediately understand the
structure of the project, and pipe the compendium to environments that
expect the standard structure with no structural modifications. 2.
Maintain a clear separation between data, method and output. Separation
means that data is treated as read only, and that all steps in getting
from source data to output is documented. Output should be treated as
disposable, and rebuildable from the programmatic application of the
method to the data. 3. Specify the computational environment that was
used to conduct the research. This provides critical information to a
replicator on the undelrying tooling needed to support the analysis.

These principles are intended to ensure that, given a certain input (the
data) and a certain operation (the method), a deterministic output
results (the analysis). With these details out of the way, an auditor
can focus on verifying the correctness of the reasoning (i.e should we
use this theory) and of the implementation (i.e are there any errors in
the math).

The richness of the R ecosystem means that one can complete the full
research life cycle from data loading to publishing without leaving the
R ecosystem (({\textbf{???}})). Data can be loaded, cleaned,
transformed, modeled and aggregated in R. The results can be written up
in RStudio - in fact, the code and write-up can be combined into a
single text-only document using the RMarkdown format - and exported to a
wide range of academically accepted typsets and formats (such as Tex,
Microsoft Word, Markdown and HTML). Because R is scriptable, R scripts
can also be used in production environments, where results can be
periodically recomputed and handed on to some other production process.

This property dramatically simplifies the complexity of building a
compendium and the required skill-set of a replicator. For instance, a
single-platform research compendium can be assessed by anyone with
knowledge of that single environment. Multi-platform compendia need
replicators well-versed in each platform, which dramatically reduces the
pool of potential replicators.

Upon publication, code versioning tools such as Git allow the public to
view all future changes to the code or data transparently, and, if code
changes over time, to assess the impact of those changes on the result.

\subsection{Tidy Extensions to Base R}\label{tidy-extensions-to-base-r}

The open source and reproducible science principles of interoperability,
readibility, modular code and common standards have been extended into
the prcatice of data manipulation by Wickham
(\protect\hyperlink{ref-Wickham2014}{2014}). The collection of R
packages which fall under this banner are collectively known as the
`Tidyverse'. These packages provide a common, idiosyncratic syntax for
data ingestion, munging, manipulation and visualization that are
opinionated in their expectations about data structure and focused on
enabling readable code (Wickham
(\protect\hyperlink{ref-Wickham2017}{2017})). According to Wickham, tidy
data intuitively maps the meaning of a dataset to its structure -

\begin{quote}
\begin{enumerate}
\def\labelenumi{\arabic{enumi}.}
\tightlist
\item
  Each variable forms a column.
\item
  Each observation forms a row.
\item
  Each type of observational unit forms a table.
\end{enumerate}
\end{quote}

Whether there is a single optimal data structure is up for debate. What
is not up for debate is that that the adoption of a common standard for
structures allows researchers to make assumptions about the meaning of
data before engaging with the content, i.e simply knowing that data is
`tidy' means that a researcher knows that each column maps to a
variable, and each row an observation, \emph{prior to looking at the
data}. Setting of standards enables the decoupling of processes -
process (a) can produce some peice of analytical output \emph{with no
knowledge of what it will be used for}. As long as it is `tidy', and
process can consume that output on the understanding that is confirms to
the `tidy' standard and take the analysis further. In this way, the
`tidy data' concept push the Unix Way concept of `build things that talk
to other things' into the practice of structuring data for reproducible
research.

\subsection{The Potential for Reproducible Research to Address The
Replication Crisis in Anomalies
Research}\label{the-potential-for-reproducible-research-to-address-the-replication-crisis-in-anomalies-research}

\begin{itemize}
\tightlist
\item
  An argument that these approaches and tools will improve backtesting,
  both for academics and for practitioners
\item
  A proviso that none of this replaces critical thinking or good data
  hygeine. Still need to get data, scrutinize it, make judgment calls
  etc etc.
\item
  Reference the definition of a compendium
\end{itemize}

\section{\texorpdfstring{A Reference Implementation of a Data Scientific
Equity Backtester in
R\label{Backtester Documentation}}{A Reference Implementation of a Data Scientific Equity Backtester in R}}\label{a-reference-implementation-of-a-data-scientific-equity-backtester-in-r}

The problem of proprietary data, the use of the log, and the
transformation into tidy data

Restructure code along the ines of a compendium Packaging Data
analytical Work Reproducibly Using R (and Friends)

\subsection{My documentation goes
here}\label{my-documentation-goes-here}

\subsection{Features}\label{features}

\begin{itemize}
\tightlist
\item
  data versioning
\item
  multi-source
\item
  multi-index
\item
  code version control
\item
  clear procedural workflow
\item
  event-driven
\item
  tidy data
\item
  pbo
\item
  host of metrics
\item
  bias avoidance
\end{itemize}

\section{\texorpdfstring{Conclusion\label{conclusion}}{Conclusion}}\label{conclusion}

This paper has explored the various ways in which the lack of
transparency in anomalies research makes it difficult to discern
spurious results from geniune findings. This `replication crisis' has
strong analogues in other academic disciplines. We have argued that the
`reproducible science' response to this crisis in science at large has
potential to address much of the issues that bedevil anomalies
replication.

We have introduced a collection of R scripts, organised into a
compendium, that can be used to conduct anomalies research in a
transparent and reproducible way. These scripts utilize only free, open
source software and to organize data along the lines of `tidy' data.
Using plain text computer code to collect, process, structure and
analyse data represents a good approach to producing research that is
easy to reproduce.

We avoid the problem of proprietary data distribution by including
customizable data query scripts. These scripts query proprietary vendors
and save the results in a standard way. Bundling these query scripts in
a compendium enables a replicator to rebuild the dataset programatically
and non-interactively from source.

This collection of scripts make it possible to test an investment
algorithm against an index of stocks, where each stock comprises a set
of daily observations of price data plus an arbitrary number of
attributes. The scripts use the event-based backtest method (as opposed
to vectorized methods) which make it easy to avoid look-ahead bias and
to introduce non-standard data to the algorithm. Transaction cost and
slippage modelling, while rudimentary, exist and can be refined.

Using this code base as a starting point should save a research a great
deal of time in preparing stock data for backtesting, and the open
source nature of the project ensures that any researcher can comb the
operations for bugs or implement features that are not present. While it
is not expected that this code base is necessary or sufficient for
end-to-end backtesting, it represents a solid base for ongoing
development.

Katzke (\protect\hyperlink{ref-Katzke2017}{2017})

\section*{References}\label{references}
\addcontentsline{toc}{section}{References}

\hypertarget{refs}{}
\hypertarget{ref-Bache2014}{}
Bache, Stefan Milton, and Hadley Wickham. 2014. ``magrittr: A
Forward-Pipe Operator for R.''
\url{https://cran.r-project.org/package=magrittr}.

\hypertarget{ref-Baum2002}{}
Baum, Christopher F., and Selcuk Sirin. 2002. ``Why should you avoid
using point-and-click method in statistical software packages?''
\url{http://fmwww.bc.edu/GStat/docs/pointclick.html}.

\hypertarget{ref-Hou2017}{}
Hou, Kewei, Chen Xue, and Lu Zhang. 2017. ``Replicating anomalies.''
\emph{NBER Working Papers}, no. No. 23394.
doi:\href{https://doi.org/10.2139/ssrn.2190976}{10.2139/ssrn.2190976}.

\hypertarget{ref-Ioannidis2005}{}
Ioannidis, John P. A. 2005. ``Why Most Published Research Findings Are
False.'' \emph{PLoS Medicine} 2 (8). Public Library of Science: e124.
doi:\href{https://doi.org/10.1371/journal.pmed.0020124}{10.1371/journal.pmed.0020124}.

\hypertarget{ref-Katzke2017}{}
Katzke, N F. 2017. ``Texevier: Package to create Elsevier templates for
Rmarkdown.'' Stellenbosch, South Africa.

\hypertarget{ref-Koenker2009}{}
Koenker, Roger, and Achim Zeileis. 2009. ``On reproducible econometric
research.'' \emph{Journal of Applied Econometrics}.
doi:\href{https://doi.org/10.1002/jae.1083}{10.1002/jae.1083}.

\hypertarget{ref-Marwick2018}{}
Marwick, Ben, Carl Boettiger, and Lincoln Mullen. 2018. ``Packaging Data
Analytical Work Reproducibly Using R (and Friends).'' \emph{American
Statistician}.
doi:\href{https://doi.org/10.1080/00031305.2017.1375986}{10.1080/00031305.2017.1375986}.

\hypertarget{ref-Munafo2017}{}
Munafò, Marcus R, Brian A Nosek, Dorothy V.M. Bishop, Katherine S
Button, Christopher D Chambers, Nathalie Percie Du Sert, Uri Simonsohn,
Eric Jan Wagenmakers, Jennifer J Ware, and John P.A. Ioannidis. 2017.
``A manifesto for reproducible science.''
doi:\href{https://doi.org/10.1038/s41562-016-0021}{10.1038/s41562-016-0021}.

\hypertarget{ref-RCoreTeam2018}{}
R Core Team. 2018. ``R: A Language and Environment for Statistical
Computing.'' Vienna, Austria. \url{https://www.r-project.org/}.

\hypertarget{ref-RStudioTeam2015}{}
RStudio Team. 2015. ``RStudio: Integrated Development Environment for
R.'' Boston, MA. \url{http://www.rstudio.com/}.

\hypertarget{ref-Stodden2013}{}
Stodden, V, D H Bailey, J Borwein, R J Leveque, W Rider, and W Stein.
2013. ``Setting the Default to Reproducible Reproducibility in
Computational and Experimental Mathematics.'' In \emph{ICERM Workshop},
19. \url{http://www.davidhbailey.com/dhbpapers/icerm-report.pdf}.

\hypertarget{ref-Wickham2014}{}
Wickham, Hadley. 2014. ``Tidy Data.'' \emph{Journal of Statistical
Software}.
doi:\href{https://doi.org/10.18637/jss.v059.i10}{10.18637/jss.v059.i10}.

\hypertarget{ref-Wickham2017}{}
---------. 2017. ``tidyverse: Easily Install and Load the 'Tidyverse'.''
\url{https://cran.r-project.org/package=tidyverse}.

% Force include bibliography in my chosen format:

\bibliographystyle{Tex/Texevier}
\bibliography{Tex/ref}





\end{document}
